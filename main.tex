
\documentclass{article}
\usepackage{natbib}

\begin{document}
\begin{flushleft}
    Maluki Muthusi \\
    P15/81741/2017
\end{flushleft}

\section*{SSL Competiting Technologies}

SSL technology has emerged as winner over other technologies that were developed to secure the web. SSL was developed at Netscape by a team lead by Teher Elgamal. There has been various versions of SSL including SSLv1 which was never released for commercial use, SSlv2, SSLv3. Beginning from SSLv3.1, SSL renamed to TLS. The naming was changed due to how SSL had evolved since version 1. \citep*{ellenmessmer}

Secure Electronic Transactions was an early communications protocol used by e-commerce websites to secure electronic debit and credit card payments. It was developed by the SET Consortium. Which consisted of Visa, Mastercard, GTE, IBM, Microsoft, Netscape, SAIC, Terisa systems, RSA and VeriSign, \citep*{williamstallings}

Secure electronic transaction was used to facilitate the secure transmission of consumer card information via electronic portals on the internet. Secure electronic transaction protocols were responsible for blocking out the personal details of card information, thus preventing merchants, hackers, and electronic thieves from accessing consumer information.

SET was not successful because of the various implementations by the vendors. Corporates feared to adopt it because of getting charged by the vendors who implemented it. The implementations were incompatible to each other, and it was expensive.

Private Communication Protocol was a protocol developed by Microsoft. It was intended to address security flows in SSL version 2. It is no longer supported and maintained and has been deprecated by Microsoft

\nocite{*}
\bibliographystyle{agsm}
\bibliography{main}
\end{document}